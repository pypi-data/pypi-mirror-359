\section*{General Information}

This task sheet consists of \pageref*{lastpage} pages (not counting the cover), numbered from 1 to \pageref*{lastpage}. Check if the sheet is complete.

\subsection*{Input}

\begin{itemize}
  \item The input must be read from standard input.

	\item All lines in the input, including the last one, end with a line break character (\texttt{\textbackslash n}).

	\item When the input contains multiple space-separated values, there is exactly one whitespace character between two consecutive values on the same line.
\end{itemize}

\subsection*{Output}

\begin{itemize}
	\item The output must be written to standard output.

	\item The output must follow the format specified in the problem statement. The output must not contain any extra data.

	\item All lines in the output, including the last one, must end with a line break character (\texttt{\textbackslash n}).

	\item When a line in the output displays multiple space-separated values, there must be exactly one whitespace character between two consecutive values.

	\item When an output value is a real number, use at least the number of decimal places corresponding to the precision required in the problem statement.
\end{itemize}

% \subsection*{Interactive Problems}

% The contest may contain interactive problems. In this type of problem, the input data provided to your program may not be predetermined but is instead specifically constructed for your solution. The judge writes a special program (the interactor), whose output is transferred to your solution's input, and your program's output is sent to the interactor's input. In other words, your solution and the interactor exchange data and may decide what to print based on the communication history.

% When writing a solution for an interactive problem, it is important to remember that if you print any data, it may first be stored in an internal \textit{buffer} and not immediately transferred to the interactor. To avoid this situation, you must use a special \textit{flush} operation every time you print data. These flush operations are available in the standard libraries of almost all programming languages:

% \begin{itemize}
% 	\item \texttt{fflush(stdout)} in \texttt{C}.
% 	\item \texttt{cout.flush()} in \texttt{C++}.
% 	\item \texttt{sys.stdout.flush()} in \texttt{Python}.
% 	\item \texttt{System.out.flush()} in \texttt{Java}.
% \end{itemize}
